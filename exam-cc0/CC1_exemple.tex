\documentclass[12pt]{article}


\usepackage[margin=2cm]{geometry}
\usepackage{array, xcolor}
\usepackage{amsthm, amssymb, amsmath}
\usepackage[english, russian]{babel}
\usepackage[utf8]{inputenc}
%\usepackage[utf8x]{inputenc}
\usepackage{graphicx}
\usepackage{ragged2e}
\usepackage{xcolor}
\usepackage[inline]{enumitem}
\usepackage{paralist}
\usepackage{fancyhdr}
\usepackage{tikz}
\usepackage{ marvosym }

\usepackage[colorlinks = true,
            linkcolor = blue,
            urlcolor  = blue,
            citecolor = blue,
            anchorcolor = blue]{hyperref}
\graphicspath{ {../} }

\definecolor{lightgray}{gray}{0.8}
\newcolumntype{L}{>{\raggedleft}p{0.20\textwidth}}
\newcolumntype{R}{p{1.0\textwidth}}
\newcommand\VRule{\color{lightgray}\vrule width 0.5pt}

\newenvironment{ttt}{\ttfamily}{\par}

\newtheorem{sol}{Подсказки к решению}

\newtheorem{zad}{Question}
\newenvironment{zad*}
  {\renewcommand\thezad{\arabic{zad}\rlap{$^\star$}$\,$}\zad}
  {\endzad}
\newenvironment{zad**}
{\renewcommand\thezad{\arabic{zad}\rlap{$^{\star\star}$}$\,\ $}\zad}
  {\endzad}



\newtheorem{Zad}{Question}
\newenvironment{Zad*}
  {\renewcommand\theZad{\arabic{Zad}\rlap{$^*$}$\,$}\Zad}
  {\endZad}
\newenvironment{Zad**}
{\renewcommand\theZad{\arabic{Zad}\rlap{$^{**}$}$\,\ $}\Zad}
  {\endZad}

\newtheorem{defi}{Определение}
\newtheorem{rem}{Замечание}

%\title{Statistiques géométriques (1er contrôle continu)}
%\date{23 octobre 2024}


%\newcommand{\cross}{\ensuremath{\square}}  %use to disable answers
\newcommand{\cross}{\ensuremath{\boxtimes}} %use to enable answers


\begin{document}

%\maketitle

 \pagestyle{fancy}
 \fancyhf{}
 \rhead{Prof. J.-B. Caillau, Prof. Kh. Kozhasov}
 \lhead{SPEM201 - ECUE MATH \quad ``$\,$Calculus II\,''}
 \lfoot{\href{https://univ-cotedazur.fr/}{Université Côte d'Azur}\\
 Laboratoire Jean Alexandre Dieudonné}
\rfoot{\vspace{0cm}\includegraphics[width=0.3\linewidth]{UCA.jpg}}

\begin{center}
  {\large  Contrôle Continu I}
  \medskip
  
{\bf \textcolor{red}{VERSION EXEMPLAIRE ET RÉDUITE.}}

   {\bf \textcolor{red}{L'examen comportera des problèmes de type et de difficulté similaires.}}
\end{center}
% {\bf Le texte de référence est en anglais.}

\smallskip
{\bf Plusieurs réponses correctes sont possibles. Les mauvaises réponses sont pénalisées.}
\smallskip
% \begin{zad}
%   Calculer le produit $z_1 z_2$ de deux nombres complexes $z_1=1+i$ et $z_2=2-i$.
%   \begin{center}
%     \begin{enumerate*}[label=$\square$, itemjoin=\qquad]
%     \item $3$
%     \item $2$
%     \item[\cross] $3+i$
%           \item $1+3i$
%           \item $1+i$
%           \item 
%     \end{enumerate}
%   \end{center}
% \end{zad}

\begin{zad}
    Calculer le quotient $z_1/z_2$ de deux nombres complexes   $z_1 = 1+i$ et $z_2 = 2-i$.
  \begin{center}
    \begin{enumerate*}[label=$\square$, itemjoin=\qquad]
    \item $\displaystyle \frac{1+3i}{5}$
    \item[\cross] $\displaystyle \frac{3+i}{5}$
    \item $\displaystyle \frac{1-i}{3}$
    \item $\displaystyle 3+i$
    \item $\displaystyle \frac{3-i}{5}$
    \end{enumerate*}
  \end{center}
\end{zad}

\begin{zad}
  Calculer $z^{-3}$ pour le nombre complexe $z=\frac{\sqrt{2}}{2}+\frac{\sqrt{2}}{2}i$.
  
  \begin{center}
    \begin{enumerate*}[label=$\square$, itemjoin=\qquad]
    \item[\cross] $-z$
    \item $\frac{\sqrt{2}}{2}-\frac{\sqrt{2}}{2}i$
    \item $-\frac{\sqrt{2}}{2}+\frac{\sqrt{2}}{2}i$
    \item[\cross] $-\frac{\sqrt{2}}{2}-\frac{\sqrt{2}}{2}i$
    \item[\cross] $e^{i\frac{5\pi}{4}}$
    \end{enumerate*}
  \end{center}
\end{zad}

\begin{zad}
Lesquelles des expressions suivantes donnent l'inverse du nombre complexe $z=2-3i$?
  \begin{center}
    \begin{enumerate*}[label=$\square$, itemjoin=\qquad]
  \item[\cross] $\displaystyle \frac{2+3i}{13}$
  \item[\cross] $\displaystyle \frac{-3+2i}{13i}$
  \item $\displaystyle \frac{2-3i}{13}$
  \item $\displaystyle \frac{2+3i}{2-3i}$
  \item[\cross] $\displaystyle \frac{2+3i}{|2-3i|^2}$
    \end{enumerate*}
  \end{center}
\end{zad}

% \begin{zad}Décrivez l'ensemble des nombres complexes qui, dans le plan complexe, correspondent aux points du cercle de rayon unité et de centre $3$.
%   \begin{enumerate}[$\square$]
%    \item $\{\,z\in\mathbb C : |z|=1\,\}$
%   \item[\cross] $\{\,z\in\mathbb C : |z-3|=1\,\}$
%   \item $\{\,z\in\mathbb C : |z+3|=1\,\}$
%   \item $\{\,z\in\mathbb C : |z-1|=3\,\}$
%   \item[\cross] $\{x+iy\in\mathbb C : x, y\in \mathbb{R},\ (x-3)^2+y^2=1 \,\}$
%       \end{enumerate}
% \end{zad}

\begin{zad}Identifier tous les nombres complexes $z\in \mathbb{C}$ telles que $\vert z\vert\leq 1$ ($\vert z\vert=$  module de $z$):\bigskip
  
  \begin{enumerate*}[label=$\square$, itemjoin=\qquad]
  \item[\cross] $\cos(2)+i\sin(2)$
  \item $z^4$, où $z=1.5 \,e^{i\frac{\pi}{2}}$ 
  \item[\cross] $-i$
  \item $\sqrt{2}$
  \item $0.9 + 0.5 i$
  \end{enumerate*}\medskip
\end{zad}

% \begin{zad}
%   Identifier toutes les paires $(z,\bar{z})$, où $\overline{z}$ désigne le conjugué de $z$:
%   \begin{enumerate}[$\square$]
%   \item $(2+9i, -2-9i)$
%   \item[\cross] $(12 e^{i\frac{3\pi}{4}}, 12 e^{i\frac{5\pi}{4}})$
%   \item[\cross] $(x-i5, x+i5)$, où $x$ est un nombre réel.
%   \item[\cross] $(\sqrt{2}\cos\left(\frac{\pi}{5}\right)+i\sqrt{2}\sin\left(\frac{\pi}{5}\right), \sqrt{2}e^{-i\frac{\pi}{5}})$
%   \item[\cross] $((1+i)^2, -2i)$
%   \end{enumerate}
% \end{zad}

% \begin{zad}Cinq nombres complexes $z_1, \dots, z_5$ sont représentés par des points dans le plan complexe. Sélectionnez tous les nombres avec $\textrm{\normalfont{Re}}(z)>0$ et $\textrm{\normalfont{Im}}(z)<0$.\medskip
%   \begin{center}
%     \begin{enumerate*}[label=$\square$, itemjoin=\qquad]
%     \item[\cross] $z_1$
%     \item $z_2$
%     \item[\cross] $z_3$
%     \item $z_4$
%     \item $z_5$
%     \end{enumerate*}\vspace{1cm}
    
%   \begin{tikzpicture}[scale=3.5, >=stealth]
%   \draw[->, thick] (-1.2,0) -- (1.2,0) node[right] {$\textrm{\normalfont{Re}}$};
%   \draw[->, thick] (0,-1.2) -- (0,1.2) node[above] {$\textrm{\normalfont{Im}}$};

% \draw[step=0.5, gray!30, very thin] (-1.2,-1.2) grid (1.2,1.2);
  
% \fill (0,0) circle (0.015);
%   \node[font=\scriptsize, anchor=south east] at (0,0) {$0$};

%   \foreach \k/\x/\y in {
%   1/0.73/-0.18,
%     2/-0.41/0.92,
%     3/0.05/-0.64,
%     4/-0.88/-0.27,
%     5/0.33/0.11
%   }{
%     \fill (\x,\y) circle (0.015);
%     \node[font=\normalsize, anchor=west] at (\x,\y) {$z_{\k}$};
%   }
% \end{tikzpicture}
%   \end{center}

% \end{zad}

% \begin{zad}
%   Parmi les nombres complexes suivants, lesquels sont solutions de l'équation $$z^3-(2+3i)z^2+(-1+5i)z+(2-2i)=0\ ?$$
  
%   \begin{center}
%     \begin{enumerate*}[label=$\square$, itemjoin=\qquad]
%     \item $3$
%     \item $-1$
%     \item[\cross] $2i$
%     \item[\cross] $1+i$
%     \item[\cross] $1$
%     \end{enumerate*}
%   \end{center}
% \end{zad}
% \newpage

\begin{zad}
  Choisissez toutes les équations qui ont (à la fois) $1$ et $i$ parmi leur solutions:
  \begin{enumerate}[$\square$]
  \item[\cross] $z^3-i z^2-(1+i)z+i=0$
  \item[\cross] $z^5-2z^4+(2+2i)z^3-(2+2i)z^2+(1+2i)z-2=0$
  \item $z-1=0$
  \item[\cross] $2z^2-(2+2i)z+2i=0$
  \item[\cross] $z^4-(2+i)z^3+(1+2i)z^2+(2-2i)z-2i=0$
  \end{enumerate}
\end{zad}

% \begin{zad}
%   Parmi les nombres complexes suivants, lesquels sont des racines 5èmes de l'unité ?
%   \begin{enumerate}[$\square$]
%   \item[\cross] $\cos\frac{2\pi}{5}+i\sin\frac{2\pi}{5}$
%   \item[\cross] $e^{i\frac{16\pi}{5}}$
%   \item[\cross] $1$
%   \item[\cross] $\bar{z}$, où $z=\cos\frac{4\pi}{5}-i\sin\frac{4\pi}{5}$
%   \item $e^{i\frac{3\pi}{5}}$
%   \end{enumerate}
% \end{zad}

% \begin{zad}
%   Quelles affirmations concernant les nombres complexes sont vraies ?
%   \begin{enumerate}[$\square$]
%   \item L'équation $z^2+z+1 = 0$ admet une solution dans $\mathbb{R}$.
%   \item Pour deux nombres complexes $z_1=r_1e^{i\phi_1}$ et $z_2=r_2e^{i\phi_2}$  écrits en coordonnées polaires on a  : $z_1=z_2$ si et seulement si $r_1=r_2$ et $\phi_1=\phi_2+2\pi$.
%   \item[\cross] Le produit d'un nombre complexe et de son conjugué est un nombre réel.
%   \item[\cross] L'inverse et le conjugué d'un nombre complexe ont le même argument (mod un multiple de $2\pi$).
%   \item Le module d'un nombre complexe ne peut pas être égal à zéro.
%   \end{enumerate}
% \end{zad}

\begin{zad}
  Quelles affirmations concernant les nombres complexes sont vraies ?
  \begin{enumerate}[$\square$]
  \item[\cross] L'équation $z^2-2z+1 = 0$ admet une solution dans $\mathbb{R}$.
  \item[\cross] Pour deux nombres complexes $z_1=x_1+iy_1$ et $z_2=x_2+iy_2$  écrits en coordonnées cartésiennes on a  : $z_1=z_2$ si et seulement si $x_1=x_2$ et $y_1=y_2$.
  \item Pour un nombre complexe $z\in \mathbb{C}$, il existe un autre nombre complexe $w\in \mathbb{C}$ tel que $zw=1$.
  \item[\cross] La somme des arguments d'un nombre complexe $z\in \mathbb{C}$ et de son conjugué $\overline{z}$ est un multiple entier de $2\pi$.
  \item Le module d'un nombre complexe est toujours positif.
  \end{enumerate}
\end{zad}

% \begin{zad}
%   Quelles affirmations concernant les EDO du premier ordre sont vraies ?
%   \begin{enumerate}[$\square$]
%   \item[\cross] La solution générale de $y'\cdot\frac{1}{y}=a(x)$ est donnée via $y(x)=y_0\exp\left(\int_{x_0}^x a(t)dt\right)$, $x_0$ et $y_0$ sont des constants réelles.
%   \item La somme de deux solutions d'une EDO linéaire du 1er ordre est également une solution.
%   \item Si $y$ est une solution d'une EDO du premier ordre, alors $-y$ est aussi une solution. 
%   \item[\cross] Le principe de superposition permet de réduire une EDO linéaire avec un second membre à une EDO sans seconde membre.
%   \item[\cross] La méthode de variation de la constante est utilisée pour résoudre des EDO linéaires du 1er ordre.
%   \end{enumerate}
% \end{zad}

\newpage
\begin{zad}
  Quelles affirmations concernant les EDO du premier ordre sont vraies ?
   \begin{enumerate}[$\square$]
  \item[\cross] L'EDO $y'=a(x)y$ admet une infinité de solutions.
  \item[\cross] La somme de deux solutions d'une EDO linéaire du 1er ordre sans second membre est également une solution.
  \item[\cross]  Si $y$ est une solution d'une EDO linéaire du 1er ordre, alors $2y$ n'est pas forcement une solution. 
  \item Le principe de superposition donne une solution sous forme fermée d'une EDO linéaire du premier ordre.
  \item[\cross] La méthode de variation de la constante permet de trouver une solution particulière de l'équation $y'+(\sin x) y - x=0$.
  \end{enumerate}
\end{zad}

% \begin{zad}
%   Lesquelles des fonctions suivantes résolvent l'équation $y'=2x$, $x\in \mathbb{R}$?
%   \begin{enumerate}[$\square$]
%   \item[\cross] $y(x)=x^2+23$
%   \item $y(x)=2x$
%   \item[\cross] $y(x)=x^2$
%   \item $y(x)=e^{2x}$
%   \item $y(x)=2x+c$, où $c$ est une constante.
%   \end{enumerate}
% \end{zad}

% \begin{zad}
%   Lesquelles des fonctions suivantes résolvent l'équation $y'+2y=e^x$, $x\in \mathbb{R}$?
%   \begin{enumerate}[$\square$]
%   \item $y(x)=\frac{1}{3}e^{2x}$
%   \item $y(x)=\frac{1}{3} e^{-2x}+e^x$
%   \item[\cross] $y(x)=e^{-2x}+\frac{1}{3}e^x$
%   \item[\cross] $y(x)=\frac{1}{3}e^{-2x}+\frac{1}{3}e^x$
%   \item[\cross] $y(x)=\frac{1}{3}e^{-2x}(3+e^{3x})$
%   \end{enumerate}
% \end{zad}

\begin{zad}
  Lesquelles des fonctions suivantes satisfont à l'équation $y'(x)+\frac{2}{x}y(x)=x e^{x}$, $x>0$? 

\begin{enumerate}[$\square$]
  \item $\displaystyle y(x)=\frac{x^2}{2}e^{x}+c$, où $c$ est une constante.
  \item[\cross] $\displaystyle y(x)=\frac{x^2}{2}e^{x}+\frac{c}{x^2}$, où $c$ est une constante.
  \item $\displaystyle y(x)=x^2 e^{x}+c$, où $c$ est une constante.
  \item $\displaystyle y(x)=\frac{x}{2}e^{x}+c x^{-2}$, où $c$ est une constante.
  \item $\displaystyle y(x)=\frac{e^{x}}{x^2}+c$, où $c$ est une constante.
\end{enumerate}

\end{zad}

\begin{zad}
  Identifier les EDO linéaires:
  \begin{enumerate}[$\square$]
  \item[\cross] $y'-\frac{1}{x}y=x^2$, $x>0$
  \item $y'-2y^2=0$
  \item[\cross] $y'+(\sin x)^{10} y = \cos x$
  \item[\cross] $y'/y = f(x)$, où $f$ est une fonction.
  \item $y'+2 \cos y = 0$
  \end{enumerate}
\end{zad}

% \begin{zad}
%   Lesquelles des fonctions suivantes résolvent l'équation $y'+\frac{1}{x}y=x^2$, $x>0$ ?

%   \begin{enumerate}[$\square$]
%   \item $y(x)=y_0 \exp\left( \int_{x_0}^x \frac{1}{t}dt\right)$, où $y_0$ et $x_0$ sont des constants réelles.
%   \item $y(x)=\frac{1}{x}$
%   \item[\cross] $y(x)=\frac{x^3}{4}$
%   \item[\cross] $y(x)=\frac{x^4+1}{4x}$
%   \item $y(x)=\frac{1}{x}+x^2$
%   \end{enumerate}

  \begin{zad} Soi $a\in \mathbb{R}$. 
    La solution générale de l'équation $y'-ay=0$, $x\in \mathbb{R}$, est donnée par
    \begin{enumerate}[$\square$]
    \item $y(x)=c$, où $c$ est une constante
    \item $y(x)=e^{-ax}$
    \item $y(x)=-ax+c$, où $c$ est une constante
    \item $y(x)=c\, e^{-ax}$, où $c$ est une constante
    \item[\cross] $y(x)=c\, e^{ax}$, où $c$ est une constante
    \end{enumerate}
  \end{zad}
  
  
\end{zad}
\end{document}
