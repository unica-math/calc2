\documentclass[12pt]{article}

\usepackage[margin=2cm]{geometry}
\usepackage{array, xcolor}
\usepackage{amsthm, amssymb, amsmath}
\usepackage[english, russian]{babel}
\usepackage[utf8]{inputenc}
%\usepackage[utf8x]{inputenc}
\usepackage{graphicx}
\usepackage{ragged2e}
\usepackage{xcolor}
\usepackage[inline]{enumitem}
\usepackage{paralist}
\usepackage{fancyhdr}
\usepackage{tikz}
\usepackage{ marvosym }
\usepackage{hyperref}

\usepackage{etoolbox}
\usepackage{pgfkeys}
\usepackage{pgfmath}


%\newcommand{\cross}{\ensuremath{\square}\ }  %use to disable answers
\newcommand{\cross}{\ensuremath{\boxtimes}\ } %use to enable answers

\newcommand{\carre}{\ensuremath{\square}\ }

% code for generating a random permutation
\newcounter{randomListLength}%   current length of our random list
\newcounter{randomListPosition}% current list index
\newcounter{newRandomListElementPosition}% position to insert new element
% insert #1 into the next position of \newRandomList unless the position
% index \randomListPosition is equal to \newRandomListElementPosition in
% which case the \newRandomListElement is added first
\newcommand\randomlyInsertElement[1]{%
  \stepcounter{randomListPosition}%
  \ifnum\value{randomListPosition}=\value{newRandomListElementPosition}%
    \listxadd\newRandomList{\newRandomListElement}%
  \fi%
  \listxadd\newRandomList{#1}%
}
% \randomlyInsertInList{list name}{new list length}{new element}
\newcommand\randomlyInsertInList[3]{%
  \pgfmathparse{random(1,#2)}%
  \setcounter{newRandomListElementPosition}{\pgfmathresult}%
  \ifnum\value{newRandomListElementPosition}=#2\relax%
    \listcsxadd{#1}{#3}%
  \else%
    \def\newRandomList{}% start with an empty list
    \def\newRandomListElement{#3}% and the element that we need to add
    \setcounter{randomListPosition}{0}% starting from position 0
    \xdef\currentList{\csuse{#1}}
    \forlistloop\randomlyInsertElement\currentList%
    \csxdef{#1}{\newRandomList}%
  \fi%
}

% define some pgfkeys to allow key-value arguments
\pgfkeys{/randomList/.is family, /randomList,
  environment/.code = {\global\letcs\beginRandomListEnvironment{#1}
                       \global\letcs\endRandomListEnvironment{end#1}
                      },
  enumerate/.style = {environment=enumerate},
  itemize/.style = {environment=itemize},
  description/.style = {environment=description},
  seed/.code = {\pgfmathsetseed{#1}}
}
\pgfkeys{/randomList, enumerate}% enumerate is the default

% finally, the code to construct the randomly permuted list
\makeatletter
\newcounter{randomListCounter}% for constructing \randomListItem@<k>'s

% \useRandomItem{k} prints item number k
\newcommand\useRandomItem[1]{\csname randomListItem@#1\endcsname}

% \setRandomItem{k} saves item number k for future use
% and builds a random permutation at the same time
\def\setRandomItem#1\par{\stepcounter{randomListCounter}%
       \expandafter\protected@xdef\csname randomListItem@\therandomListCounter\endcsname{\noexpand\item#1}%
       \randomlyInsertInList{randomlyOrderedList}{\therandomListCounter}{\therandomListCounter}%
}%
\let\realitem=\item
\makeatother
\newenvironment{randomList}[1][]{% optional argument -> pgfkeys
  \pgfkeys{/randomList, #1}% process optional arguments
  \setcounter{randomListLength}{0}% initialise length of random list
  \def\randomlyOrderedList{}% initialise the random list of items
  % Nthing is printed in the main environment. Instead, \item is
  % used to slurp the "contents" of the item into randomListItem@<counter>
  \let\item\setRandomItem%      
}
{% now construct the list environment by looping over the randomly ordered list
  \let\item\realitem
  \setcounter{randomListCounter}{0}
  \beginRandomListEnvironment\relax
    \forlistloop\useRandomItem\randomlyOrderedList
  \endRandomListEnvironment
}

% test compatibility with enumitem
\usepackage{enumitem}
\newlist{Testlist}{enumerate}{1} %
\setlist[Testlist]{label*=\arabic*.}
\setlist{nosep}\parindent=0pt% for more compact output

\begin{document}

\pagestyle{fancy}
 \fancyhf{}
 %\rhead{Prof. J.-B. Caillau, Prof. Kh. Kozhasov}
 %\lhead{\bf Nom :\hspace*{4.0cm} Prénom : \hspace*{5.0cm} No.\ étudiant :\hspace*{5.0cm}}
 \lhead{\bf Nom :\hfill Prénom : \hfill No.\ étudiant :\hfill}
 %\rhead{Calculus II - L1}
\lfoot{\href{https://univ-cotedazur.fr/}{Université Côte d'Azur}\\
 Département de mathématique}
\rfoot{\vspace{0cm}\includegraphics[width=0.3\linewidth]{UCA.png}}

\begin{center}
  {\large  \bf Exam CC no. 1 -- Calculus II}
  \medskip
  
%{\bf \textcolor{red}{VERSION EXEMPLAIRE ET RÉDUITE.}}

\end{center}
% {\bf Le texte de référence est en anglais.}

\begin{center}
     \textbf{Nota bene.} Plusieurs réponses correctes sont possibles. \\ Chaque réponse correcte apporte des points. Les mauvaises réponses sont pénalisées.
\end{center}

\begin{randomList}[environment=Testlist, seed=2] % seed = 1, 2 ou 3
\item Calculer le produit $z_1z_2$ des deux nombres complexes $z_1=1+i$ et $z_2=2-i$.  
\begin{center}
    \carre $3$ \quad
    \carre $2$ \quad 
    \cross $3+i$ \quad 
    \carre $1+3i$ \quad 
    \carre $1+i$\end{center}
    
\item  
Calculer $z^4$ pour le nombre complexe $z=\frac{\sqrt{3}+i}{2} \cdot$
  \begin{center}
    \cross $-\frac{1}{2}+\frac{\sqrt{3}}{2}i$\qquad
    \carre $-1$
    \qquad\carre $i$
    \qquad\carre $-i$
    \qquad\carre $\frac{1}{2}-\frac{\sqrt{3}}{2}i$
  \end{center}
  
\item 
Calculer $(1+i)(2-3i)(1-i)^{-1}$.
\begin{center}
\cross $3 + 2i$ \qquad\carre  $3 - 2i$ \qquad\carre  $2 + 3i$ \qquad\carre  
$\frac{5}{2} + 2i$ \qquad\carre $3 + i$ 
\end{center}

\item 
Les nombres complexes $z$ et 
$w$ sont donnés en coordonnées cartésiennes ($z=x+iy$, $w=u+iv$) ou en coordonnées polaires ($z=re^{i\phi}$, $w=\rho e^{i\theta}$).    Quelles expressions donnent 
$izw$ ?\\ \\
\carre $i(xu-yv) + (xv+yu)$\\
\cross $-(xv+yu) + i(xu-yv)$\\
\carre $r\rho\,e^{i(\phi+\theta)}$\\
\cross $r\rho\,e^{i(\phi+\theta+\pi/2)}$\\

\item Décrivez l'ensemble des nombres complexes qui, dans le plan complexe, correspondent aux points du cercle de rayon $1$ et de centre $3$.\\ \\
\carre $\{\,z\in\mathbb C : |z|=1\,\}$\\
\cross $\{\,z\in\mathbb C : |z-3|=1\,\}$\\
\carre $\{\,z\in\mathbb C : |z+3|=1\,\}$\\
\carre $\{\,z\in\mathbb C : |z-1|=3\,\}$\\
\cross $\{x+iy\in\mathbb C : x, y\in \mathbb{R},\ (x-3)^2+y^2=1 \,\}$\\
  
\item  
Choisissez toutes les formules correctes pour un nombre complexe non nul $z\neq 0$, représenté soit en coordonnées cartésiennes ($z=x+iy$), soit en coordonnées polaires ($z=re^{i\phi}$).\\ \\
\cross $\bar{z}^{\,-1}= \frac{z}{\vert z\vert^2}$\\
\cross $z^7=r^6 e^{8i\phi} \bar{z}$\\
\cross  $\frac{z}{\vert z\vert} = \cos \phi + i\sin \phi$\\
\carre $\bar{z} z =r^2 e^{2i\phi}$\\
\carre $i z=y+i x$\\
    
\item Choisissez toutes les expressions égales à $\cos⁡(4x)$, $x\in \mathbb{R}$.
\begin{center} 
    \cross $2\cos^2(2x)-1$ 
    \quad\cross $8\cos^4 x - 8\cos^2 x + 1$ 
    \quad\cross $\cos^2 (2x)-\sin^2 (2x)$ 
    \quad\carre $4\cos^2 x - 3$
    \end{center}
    
\item 
  Identifier toutes les paires $(z,\bar{z})$, où $\overline{z}$ désigne le conjugué de $z$.\\ \\
  \carre $(2+9i, -2-9i)$\\
  \cross $(12 e^{i\frac{3\pi}{4}}, 12 e^{i\frac{5\pi}{4}})$\\
  \cross $(x-5i, x+5i)$, où $x$ est un nombre réel.\\
  \cross $(\sqrt{2}\cos\left(\frac{\pi}{5}\right)+i\sqrt{2}\sin\left(\frac{\pi}{5}\right), \sqrt{2}e^{-i\frac{\pi}{5}})$\\
  \qquad\cross $((1+i)^2, -2i)$\\
  
\item 
Quelles expressions donnent la partie imaginaire de $(e^{i\phi})^3-3e^{i\phi}+1$, $\phi\in \mathbb{R}$ ?\\ \\
  \carre $-3\cos^3\phi$\\
  \carre $4\sin^3 \phi$\\
  \cross $\sin (3\phi)-3\sin \phi$\\
  \carre $-3\sin \phi$\\
  \carre $\cos (3\phi) - 3\cos \phi$\\
 
\item 
    Parmi les nombres complexes ci-dessous, lesquels sont solutions de l'équation  $2026 z^{2026}-2026=0$ ?
   \begin{center} 
        \qquad\cross $e^{i\pi}$ 
        \qquad\carre $2026$
        \qquad\cross $\cos\frac{\pi}{1013} + i \sin\frac{\pi}{1013}$
        \qquad\cross $e^{\frac{2\pi i}{2026}}$
        \qquad\carre $i$
    \end{center}
      
\item 
  Parmi les nombres complexes suivants, lesquels sont solutions de l'équation $$z^3-(2+3i)z^2+(-1+5i)z+(2-2i)=0\ ?$$
  \begin{center}
       \carre $3$
    \qquad\carre $-1$
    \qquad\cross $2i$
    \qquad\cross $1+i$
    \qquad\cross $1$\\
    \end{center}
    
\item 
  Parmi les nombres complexes suivants, lesquels sont des racines 5èmes de l'unité ?
 \begin{center} 
  \qquad\cross $\cos\frac{2\pi}{5}+i\sin\frac{2\pi}{5}$
  \qquad\cross $e^{i\frac{16\pi}{5}}$
  \qquad\cross $1$
  \qquad\cross $\bar{z}$, où $z=\cos\frac{4\pi}{5}-i\sin\frac{4\pi}{5}$
  \qquad\carre $e^{i\frac{3\pi}{5}}$\\
  \end{center}
  
\item 
Quelles affirmations sont vraies ?\\ \\ 
  \carre L'équation $z^2+z+1 = 0$ admet une solution dans $\mathbb{R}$.\\
  \carre Pour deux nombres complexes $z_1=r_1e^{i\phi_1}, z_2=r_2e^{i\phi_2}$  écrits en coordonnées polaires on a: $z_1=z_2$ si et seulement si $r_1=r_2$ et $\phi_1=\phi_2+2\pi$.\\
  \qquad\cross Le produit d'un nombre complexe et de son conjugué est un nombre réel.\\
  \qquad\cross L'inverse et le conjugué d'un nombre complexe ont le même argument\\ (mod un multiple de $2\pi$).\\
  \qquad\carre Le module d'un nombre complexe ne peut pas être égal à zéro.\\
  
\item 
Quelles affirmations sont vraies ?\\ \\
  \cross La solution générale de $y' = a(x)y$ est donnée via $y(x)=y_0\exp\left(\int_{x_0}^x a(t)dt\right)$, $x_0$ et $y_0$ sont des constantes réelles.\\
  \carre La somme de deux solutions d'une équa.\ diff. linéaire du 1er ordre est également une solution.\\
  \carre Si $y$ est une solution d'une équa.\ diff. du premier ordre, alors $-y$ est aussi une solution.\\
  \cross Le principe de superposition permet de réduire une équa.\ diff. linéaire avec un second membre à une équa.\ diff. sans second membre.\\
  \cross La méthode de variation de la constante est utilisée pour résoudre des équa.\ diff. linéaires du 1er ordre.\\

\item 
Lesquelles des fonctions suivantes satisfont l'équation $y'=2x$, $x\in \mathbb{R}$ ?\\ \\
   \cross $y(x)=x^2+23$\\
  \carre $y(x)=2x$\\
  \cross $y(x)=x^2$\\
  \carre $y(x)=e^{2x}$\\
  \carre $y(x)=2x+c$, où $c$ est une constante.\\

\item 
Lesquelles des fonctions suivantes satisfont l'équation $y'+2y=e^x$, $x\in \mathbb{R}$ ?\\ \\
  \carre $y(x)=\frac{1}{3}e^{2x}$\\
  \carre $y(x)=\frac{1}{3} e^{-2x}+e^x$\\
  \cross $y(x)=e^{-2x}+\frac{1}{3}e^x$\\
  \cross $y(x)=\frac{1}{3}e^{-2x}+\frac{1}{3}e^x$\\
  \cross $y(x)=\frac{1}{3}e^{-2x}(3+e^{3x})$\\

\item 
Lesquelles des fonctions suivantes satisfont l'équation $y'+\frac{1}{x}y=x^2$, $x>0$ ?\\ \\
\carre $y(x)=y_0 \exp\left( \int_{x_0}^x \frac{1}{t}dt\right)$, où $y_0$ et $x_0$ sont des constantes réelles.\\
  \carre $y(x)=\frac{1}{x}$\\
  \cross $y(x)=\frac{x^3}{4}$\\
  \cross $y(x)=\frac{x^4+1}{4x}$\\
  \carre $y(x)=\frac{1}{x}+x^2$\\

\item 
Une équa.\ diff. linéaire du premier ordre peut se présenter sous la forme :\\ \\
 \cross $y' + a(x)y = b(x)$\\
    \carre $y'' + a(x)y' + b(x)y = c(x)$\\
    \carre $(y')^2 + a(x)y = b(x)$\\
    \cross $\dfrac{dy}{dx} + p(x)y = q(x)$\\

\item  La solution générale de $y' - 3y = 0$ est ($c$ désigne une constante arbitraire) :
\begin{center}
    \cross $y = ce^{3x}$ \qquad
    \carre $y = ce^{-3x}$ \qquad 
    \carre $y = e^{3x} + c$ \qquad 
      \carre $y = e^{3x}$\qquad
      \carre $y = 3x + c$
\end{center}

\item 
Pour résoudre $y' + p(x)y = q(x)$ par méthode de variation de la constante, on cherche une solution particulière de la forme :\\ \\
   \cross $y = c(x)y_h$ où $y_h$ est une solution de l'équation homogène\\
    \carre $y = c$ où $c$ est une constante\\
    \cross $y = c(x)e^{-\int p(x)dx}$\\
    \carre $y = \dfrac{q(x)}{p(x)}$\\

\item 
    Pour l'équation $y' + y = 5$, une solution particulière est :
\begin{center}
    \qquad\cross $y = 5$
    \qquad\carre $y = e^{-x}$
    \qquad\carre $y = 5x$
    \qquad\carre $y = 0$
\end{center}

\item 
    Parmi les fonctions suivantes, lesquelles sont solutions de l'équation $y' - y = e^x$ ?
\begin{center}
    \qquad\cross $y = xe^x$
    \qquad\cross $y = xe^x + 5e^x$
    \qquad\carre $y = e^x$
    \qquad\carre $y = 2xe^x$
\end{center}

\item 
Sélectionnez toutes les paires 
\begin{center}
(équation différentielle linéaire du premier ordre,$\ $ une solution de cette équation)
\end{center}
\cross ($y' + 3y = 6e^{-x}$,
$y = 3e^{-x} + 10e^{-3x}$), $x\in \mathbb{R}$\\
\carre 
($y' + \frac{2}{x}y = x^2$, $y = \frac{2}{x^2}$), $x > 0$\\
\cross ($y' + e^{x}y = e^{x}$, $y = 1+ e^{-e^{x}}$),  $x\in \mathbb{R}$\\
\carre ($y' + (\tan x)\,y = \frac{1}{\cos x}$, $y= \cos x$), $0<x<\frac{\pi}{2}$\\
\cross
($y' + 2xy = 4x$, $y = 2$), $x\in \mathbb{R}$\\

\end{randomList}

\hspace{0.3cm}24.\  Cinq nombres complexes $z_1, \dots, z_5$ sont représentés par des points dans le plan complexe.

\qquad\ Sélectionnez tous ceux vérifiant $\textrm{\normalfont{Re}}(z)>0$ et $\textrm{\normalfont{Im}}(z)<0$.
  \begin{center}
     \qquad\cross $z_1$
    \qquad\carre $z_2$
    \qquad\cross $z_3$
    \qquad\carre $z_4$
    \qquad\carre $z_5$
  \end{center}
\begin{center}
\begin{tikzpicture}[scale=4.5, >=stealth]
  \draw[->, thick] (-1.2,0) -- (1.2,0) node[right] {$\textrm{\normalfont{Re}}$};
  \draw[->, thick] (0,-1.2) -- (0,1.2) node[above] {$\textrm{\normalfont{Im}}$};

\draw[step=0.5, gray!30, very thin] (-1.2,-1.2) grid (1.2,1.2);
  
\fill (0,0) circle (0.015);
  \node[font=\scriptsize, anchor=south east] at (0,0) {$0$};

  \foreach \k/\x/\y in {
  1/0.73/-0.18,
    2/-0.41/0.92,
    3/0.05/-0.64,
    4/-0.88/-0.27,
    5/0.33/0.11
  }{
    \fill (\x,\y) circle (0.015);
    \node[font=\normalsize, anchor=west] at (\x,\y) {$z_{\k}$};
  }
\end{tikzpicture}\end{center}
\end{document}